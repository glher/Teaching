\chapter*{If You Have To Remember Something}
\addcontentsline{toc}{chapter}{Summary} % Add the preface to the table of contents as a chapter

This book is long, and it contains a lot of information and calculation estimations. It can be hardy, at times, to sit down and organize all the ideas and insights gained into a useful view of the world and the energy transition.

This section summarizes the main point that tell the story one should care about.

\begin{itemize}
\item Energy is the economy
\item Fossil fuels were an incredible discovery for humanity, but now have us in a technological lock-in, unfortunately too early
\item Climate change will impact us. There is no avoiding it, but we can mitigate its effect by stopping our emissions and increasing our preparedness
\item The scale of the necessary transitions are staggering. Most humans are not very good at three things: thinking at scale,  probability assessment, and complex interlocking networks. Unfortunately, this issue scores high in all three categories.
\item We have the technology to solve most of the issues in theory. Multiple scenarios are possible, and we do have the scientific knowledge to know what can and cannot work.
\item A fully renewable world hits immense hurdles very quickly, because the world has very finite resources.
\item Nuclear is not desirable in a lot of places, because the world is populated with irrational humans.
\item We need to think of a transition in a sustainable way, instead of do the same thing previous generations did: "We did our best, they will have to figure it out". A one-time transition to fully renewable systems may, if the stars align, work. But we do not want to end up stuck with a system that cannot work anymore, the carbon reduction have to be long term.
\item Technological bet is not something I would recommend. Scientists may and will make breakthrough. Their scalability is all that matters, and in that regard media can be misleading.
\item The smartest way, and the best way not to fail, is to go toward a transition mix, where nuclear and renewable are complementary. Location matters.
\item Investing in one technology does not mean not investing in the other. The enemy is carbon-emitting production.
\item We have to stop fighting with one hand in the back, blindfolded, and jumping on one foot.
\end{itemize}