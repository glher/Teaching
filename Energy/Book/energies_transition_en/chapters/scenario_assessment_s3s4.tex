\setchapterpreamble[u]{\margintoc}
\chapter{Scenario Assessments}
\labch{scenario_assessments}

In this chapter I will go over the physical limitations of 100\% Renewable scenarios, from pumped storage locations and scale to batteries materials mining and photovoltaic and wind land area constraints.

This section discusses the issues for nuclear, and for a mix

\section{Locations II -- Nuclear Siting}

\blindtext

\section{Materials IV -- Uranium}

\blindtext

\section{Public Opinion}

Public opinion is difficult to change quickly and react strongly to fear

\blindtext

\section{Waste}

Waste is an issue, but not a big one. Appendix: Hopes from Oklo

\blindtext

\section{Timeline}

Nuclear takes time to develop, while renewable are fast.

\blindtext

\section{Economical mix}

Renewable need to be super cheap to compete with nuclear fuel generation in today's market, or policies will need to be created to account for this

\blindtext



\section{The Digest}

\begin{kaoboxgreen}[frametitle=Main Takeaways]

\begin{itemize}
\item The \emph{finite materials problem} is very real and is an unmovable wall in the way of 100\% renewable scenarios.
\item Renewable energy is indeed renewable by definition, but capturing that energy is not renewable.
\item Building up energy storage systems or wind turbines use up either finite geographical locations or finite materials resources at an alarming and absolutely not sustainable rate.
\item Recycling will have to become prevalent for some materials. It will not be enough to cover the needs of a 100\% renewable system, but still absolutely needs to be developed.
\end{itemize}
  
\end{kaoboxgreen}