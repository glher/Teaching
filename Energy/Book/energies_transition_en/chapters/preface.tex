\chapter*{Preface}
\addcontentsline{toc}{chapter}{Preface} % Add the preface to the table of contents as a chapter

I am of the opinion that everyone should, at some point in their life, and preferably early on, takes the time to sit down and go through a first-order estimation of a few societal problems. Climate change is probably the most important challenge humanity faces, and solving it while minimizing the human tragedies is about to become tenuous at best. While this book does not offer new paradigms or groundbreaking science, it aims at guiding people toward the most efficient -- and in this real, finite world of ours, only realistic -- path to a less extreme future. The main ideas behind this text come from two observations:

\begin{itemize}
	\item People have a tendency to pick a camp, and, often slowly, get entrenched and become less and less open in their views;
	\item The transition issue is a very complicated system, with multiple interlocking pieces and possible future predictions.
\end{itemize}

Social media is, in my opinion, a negative driver of progress at scale, at least in the context of dealing with global crises. As has been observed for different problems, it can serve as an echo chamber which creates a feedback loop reinforcing belief despite potential opposite evidence. The complexity of the system causes people to often brush asides any concerns over their views of \emph{the solution}, by using a classical argument: Technology will improve and figure it out in times. While this is certainly possible, it is not something we can afford to bet on, especially when technology today already tells us what could be done.

During my days as a Nuclear Engineer in France, I did not consider those issues. I knew that the low carbon emission of nuclear energy was a positive thing, but I did not realize how important it was. When I left that position to pursue research in energy and complex systems in the USA, I co-founded a data company, which looks at terabytes of information about the world today and the impact of climate change on its potential, in the hope of forcing investment funds to switch more of their portfolio and policies toward sustainable and efficient technology mixes. After a lot of exploration and revisiting my assumptions faced with new information, I slowly, over time, realized what the underlying issues were. After some grieving time, and numerous nights where sleep eluded me (kids may have played a role in that, mind you), I finally found a way that I believe can help people come to term with the problem and realize the issue of scale.

The goal of this book is to go over the basics to set the context, and help people reach their own conclusions, by plugging in the numbers. It is quite fundamental for you to know something more here: If you disagree with my take, with the input data, or with the calculations, that is \emph{a very good thing}. I encourage you to use the human contradictory nature to go deeper and prove me wrong. Doing so will teach you more about the problem at hand, the potential solutions and the most adequate and fair paths forward than any reading could.

It is also in my opinion important to understand that, behind the scenes, there is a scientific war raging between two camps. On one side, you have the proponents of 100\% renewable system, who say that this future is attainable economically. On the other side, the scientists who say that the assumptions used are flawed and that the focus should be on a wider mix of technologies to be certain to reach the end goal. While this book will likely, by default of its method, fall on the latter side, I would like to state two things.

The first one is that, in my experience, any theoretically optimized solution has rarely come to fruition in the real world. Whether we are talking about budget, timelines, or end-product match with the initial specification, and whether we are talking about scientific projects, kids education, policy implementations, or personal projects, things tend to evolve with time in the real world. This may sometimes go in the right direction and help you finish a project in time, under budget, with all the specifications met. However, I find that this effect often results in compromises in a negative direction. In that regard, I consider that an optimized plan needs to be validated by first order approximation, or the risk of just one thing derailing the entire plan is really high.

The second thing is that if you do not have a unified scientific community on a given issue, you will not have a unified world tackling the issue. Look at climate change itself. Despite the vast majority of scientists, and the entirety of competent ones, raising the alarm and acknowledging the impact of this issue on the real world, you still have a non negligible portion of populations who do not follow suits and do not trust the experts. It is in my opinion a fact of human society that if even the scientific community is divided, the population at large can only be even more strongly in conflicting positions, and social media will not help that.

The first chapter of this book is introductory and covers the basics of what you can expect and not expect. Next, we discuss how fundamental energy is, and has always been, to human society. The marvel of fossil fuels are considered. The second part deals with the consequences of our rapid expansion, both in terms of population and (selective, not well distributed) comfort. It approaches the problem of climate change and the severe impacts we are already on our way to see.

In the third part, I give some quick reminders on the energy sources at our disposal. Do not expect great details, only a way to ensure that everyone is one the same page.

I started writing this book as a thought experiment, to find a way to quantify this complex system using simple maths and orders of magnitude. My experience, as well as plenty of smarter people than me, have taught me that if a complex intricate model does not approximately match a simple estimate correctly done, the error lie in all likelihood in the complex model. Assumptions are extremely important, and some can be, by their very definition, contradictory. A complex model tend to miss the mark on the main assumptions by having to account for or optimize too many things simultaneously.

This is why, in the fourth part, we will go through an entire simple case study of different scenarios transition in various representative countries. Starting from the energy needs to decarbonize, and using reasonable assumptions for technology costs, we will compare several scenarios. Most notably, we will assess the possibility, advanced by many, of having a 100\% renewable grid. We will then spend some time pointing the very important flaws in the various scenarios.

In the last part, we will ponder on what we have seen and try to see what the best path forward is. Being united in our game plan is of the utmost importance to effectively tackling the issue at a global scale. All is not lost, but people need to realize that \emph{we will feel the impact of climate change}, even if we were to go to net zero literally overnight. The question is, how much are we going to feel it, and what can we do to mitigate the impacts on societies throughout the globe at best we can.

\begin{flushright}
	\textit{Guillaume L'Her}
\end{flushright}
