\setchapterpreamble[u]{\margintoc}
\chapter{Scenarios Case Study}
\labch{scenario_computation}

In this chapter I will go over computing the amount of capacity to install, the amount of storage to install, the impact to expect on the grid, and the resulting values given for each scenarios


\begin{table}[ht]
\caption[Input Data and Notations]{Input Data and Notations}
\labtab{input_notations}
\begin{tabular}{ c c c }
	\toprule
	Data & Notation & Units\\
	\midrule
	Annual Energy Production Needs & $E_a$ & TWh \\
	Installed Capacity & $P_i$ & GW\\
	Technology Load Factor & $f$ & \% \\
	Peak Production & $p_{max}$ & \% \\
	Energy Storage Capacity & $E_s$ & TWh\\
	Power Storage Capacity & $P_s$ & GW \\
	Power Excess & $P_e$ & GW \\
	Storage Round-Trip Efficiency & $e$ & \% \\
	Installed Capacity with Storage & $P_{i,s}$ & GW \\
	Minimum Yearly Need & $P_{min}$ & GW \\
	Energy Storage Fraction & $f_s$ & \% \\
	Installation Costs & $C_i$ & \$ \\
	Storage Costs & $C_s$ & \$ \\
	Dismantling Costs & $C_d$ & \$ \\
	Grid Costs & $C_g$ & \$ \\
	Duration of Storage need & $t$ & hours \\
	\bottomrule
\end{tabular}
\end{table}


\section{How Many GigaWatts?}

We have seen from \vrefch{transition_needs} the amount of energy to be produced annually, in TWh, per country of interest. In that chapter we also derived the load factor per technology, in the latest years.


An important factor to consider now is the capacity of electricity you need to install in order to obtain this necessary energy produced at the end of any given year. This is shown in equation~\ref{eqn_power_installed}.

\begin{remark}
In order to obtain the energy generated over a year in Wh, one needs to multiply the power installed by the number of hours in a year\sidenote[][-2mm]{Mind the units! Recall that $E_a$ is in Terrawatt-hour, which is a thousand Gigawatt-hour and a billion kilowatt-hour. So, the $P_i$ obtained will be in Terrawatt. We are most used to dealing with installed capacity in Gigawatt, which implies a multiplication by $10^{-3}$ of your result}.

\begin{equation}\label{eqn_power_installed}
P_i = \frac{E_a}{f * 8760}
\end{equation}

\end{remark}

\section{Storage Requirements}



\begin{kaobox}[frametitle=What we have said]
Wind and Solar can be down or low for extended periods of time, during which batteries have to pick up the slack. This represents $x$ TWh over the year and gives us a battery output required assuming a duration period to cover of $y$ days.

While this may seem to not take advantage of a combining effect, this is actually a reasonable assumption in the real, non-perfectly-optimized, world. In such optimized modeling simulations, real world constraints such as geopolitics and social difficulties are often ignored or misrepresented.

\end{kaobox}







\section{Grid impact}

We are making the assumption here that the grid is able to deal with a lot of excess energy, or that this is used for non-time-dependent processes like hydrogen production, desalination, or other. This comes with grid requirements which can be very difficult to meet.

Grids have been designed to accommodate for around the power capacity that is currently installed, in a mostly centralized, optimized, system. In order to move to a 100\% renewable system, hence mostly decentralized, non-optimized, system, grid developments will be necessary, and transmission lines will have to be built. In terms of cost, it is no secret that it is a lot more expensive to install 500 lines of 100 MW (accommodating an installed capacity of 50 GW) than 25 lines of 2 GW (accommodating the same installed capacity). Materials, public works, distance covered, all those and more factor into the increased price.

More importantly, the installed capacity being greater, the grid has to be reinforced to account for the maximum expected output. While the load factor over the year is relatively low for wind or solar, you will have peaks throughout the year at up to 70\% and 90\% of the installed capacity respectively\sidenote[][-2mm]{A sunny day at noon, a windy day on all your wind farms, and all of the sudden you have way more energy than you know what to do with!}. 

You are presented with three basic options for this energy.
\begin{itemize}
\item Store it
\item Use it
\item Dump it
\end{itemize}

The first option is to store it for future use as long as you have the installed storage power capacity. Given your requirements, this has to be done in priority. The second option is to use that excess energy for auxiliary uses, such as hydrogen production. This is not as simple as it seems, and is unfortunately not a magical solution, though it has a lot of merits. The third option is to accept the loss and curtail your production.


\begin{remark}
Consequently, the grid should be ready to deal with an excess power up to what is given in the following equation.

\begin{equation}\label{excess_power}
P_e = P_i * p_{max} - P_{min}
\end{equation}
\end{remark}

\section{Scenarios Results - Dollar Costs}

First, we calculate an estimate of the costs per scenario, for each country, in \vreftab{scenario_costs_estimates}. This shows that things get very expensive, and that some options are clearly more efficient economically than others. \vreftab{scenario_costs_ratio} shows the normalization relative to the least expensive scenario each time\sidenote[][-2mm]{Nuclear, with a score of 1}. Even though in our case studies we do not vary the prices locally, and assume that a project will cost the same in every country, we see that the ratios vary. This is mostly due to the effect of changing load factors. Not every country is equal when it comes to natural resources, and wind and solar are natural resources.


\begin{table}[ht]
\caption[Scenarios 1 Costs Estimate over 100 years per country (Trillions US Dollars -- Electricity only -- 7 days of storage)]{Scenario Costs Estimate over 100 years per country (Trillions 2020 US Dollars) -- Electricity only -- 7 days of storage}
\labtab{scenario_costs_estimates}
\begin{tabular}{ c c c c c c c }
	\toprule
	Country & S1a & S1b & S1c & S1d & S1e & S1f \\
	\midrule
	France   & 2.5 & 20.6 & 3.9 & 22.0 & 2.9 & 21.0 \\
	USA   & - & - & - & - & - & - \\
	Brazil   & - & - & - & - & - & - \\
	China  & - & - & - & - & - & - \\
	Nigeria  & - & - & - & - & - & - \\
	Nigeria\sidenote[$\dagger$][-2mm]{With access to an average French person electricity}  & - & - & - & - & - & - \\
	\bottomrule
\end{tabular}
\end{table}

\section{Scenarios Results - CO2 saved by country}

\blindtext

\section{The Digest}


\begin{kaoboxgreen}[frametitle=Main Takeaways]

\begin{itemize}
\item We derive the main simple equations to be used to approximate the costs of a scenario
\item The capacity to install per scenario is given depending on a constant need assumption
\item The storage consequence is approached, and we go easy on it
\end{itemize}
  
\end{kaoboxgreen}