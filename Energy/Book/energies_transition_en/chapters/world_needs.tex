\setchapterpreamble[u]{\margintoc}
\chapter{The World Needs}
\labch{world_needs}

In this chapter I scale the renewables scenario to the world, with current assumptions and social assumptions


We are under severe time constraints, and it is important to consider, from what we talked about previously, that continued economic growth is not a fact of nature, and is extremely dependent on energy. Peak oil is approaching quickly, and some countries have passed it already, which notably means that it is likely any energetic transition will have to take place in a recessive context. Investment will not be available in the same way. Conflicts, devastation, migrations on a scale never seen before may also happen due to the climate change very impact, with much stronger hazards, as discussed in a previous section.

We have shown that what is needed, by default of being the only option, is nuclear and renewables, and energy storage for localized use only.

Nuclear “reinstallation” will take time, and we can use that time to ramp up to a good penetration of renewable and make progress. Something is better than nothing, as long as there is a plan.

\blindtext


\section{The Digest}


\begin{kaoboxgreen}[frametitle=Main Takeaways]

\begin{itemize}
\item This has not been done yet
\item Reading this will teach you absolutely nothing
\item I am serious, I could type random letters and it would give you as much information
\item Fedhiz gavartz hedtz inewps
\end{itemize}
  
\end{kaoboxgreen}