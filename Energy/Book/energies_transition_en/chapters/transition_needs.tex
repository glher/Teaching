\setchapterpreamble[u]{\margintoc}
\chapter{The Transition Needs}
\labch{transition_needs}

In this chapter I will assess two main scenarios. The first one assumes that the world of 2050 and later looks very similar to the world of today. That is, energy poverty, and by extension poverty, is not solved and the needs do not increase. We even look at an optimistic energy need, where the fossil fuels heat conversion is replaced by a more effective use of electricity, lowering the energy needs. A second scenario is also shown where we consider everyone on the planet to have a reasonable access to energy, with every country consuming at the Eastern European rate.

\begin{kaobox}[frametitle=Transition Scenarios]
\begin{itemize}
\item Scenario EOSP

The scenario Energy Optimist, Society Pessimist consider that the current energy needs only are transitioned. This means that the billions of people currently in energy poverty stay in that position for the next century.

\item Scenario EPSO

The scenario Energy Pessimist, Society Optimist consider that the world will see the electrification of most people living in poverty today, at a level  around 30,000 kWh or energy consumed per year per capita. As a reference point, this is roughly the situation of Poland, Greece or Portugal.
\end{itemize}
\end{kaobox}

Scenario EOSP indicates the need to transition 160,000 TWh, while scenario EPSO would have us transition 200,000 TWh.


\todo{Clarify GWe versus GWth}


When the media discuss energy transition, I feel that they often misrepresent, or don't explain enough, what they mean. There is a very large difference between electricity and energy. Electricity encompasses only the electric power uses, such as what allows a home outlet to charge a phone, plug a fridge, etc. Energy consists of all of the electricity as well as transportation and heating.

When using gas to power a car or to heat a location, a lot of the energy is lost in the process. Electricity would allow you to be more efficient. Consequently, the electrification of the transportation and thermal sector implies a better efficiency and less consumption, translating to a lower power needed.

In the following, we will make two things distinct:


\begin{kaobox}[frametitle=Electricity Transition]
Electricity transition is the removal of all energy from fossil fuels from the electrical grid. In other words, the gasoline powered cars can stay, the coal and gas plants have to go.
\end{kaobox}

\begin{kaobox}[frametitle=Energy Transition]
Energy transition is the removal of all energy from fossil fuels, to be replaced with non carbon-emitting sources.
\end{kaobox}

Electricity transition is not energy transition.


\section{Current Energy Sources}

The data for the current energy generation mix in the world is obtained.

[Placeholder for energy donut graph]

[Placeholder for energy time series graph]

[Placeholder for zoom on renewable share graph]



\todo{Approximately two-thirds of this feeds heat engines (power plants, cars, etc.), at an average efficiency of 30\%, delivering 0.6 TW of useful work in the bargain. The other (approximate) third is direct heat (lots of this in industrial process heat), and electricity from nuclear and hydro sources. Imagining that we replace our heat engines with direct electricity and electrified transport, we need less total power, accounting for some inefficiency.}

Assuming an extremely optimistic electrification of the world transport sectors and other heat engines, let us assume that only 150000 TWh of energy will be needed every year. Note that this is probably on the very, very low end of future needs.

\section{Current Carbon Emissions}

This section computes an estimate of the yearly carbon emissions in both scenarios assuming a business-as-usual situation. This will give us a good idea of what we need to avoid producing.


\section{Scenarios Pathway}

This section looks at all the installation we need for each scenario. It should be a table:



\begin{table}[ht]
\caption[Technology capacity per Scenario S1]{Technology capacity per Scenario S1}
\labtab{technology_capacity_scenarios_s1}
\begin{tabular}{ c c c c c c c }
	\toprule
	Technologies & S1a & S1b & S1c & S1d & S1e & S1f \\
	\midrule
	Solar & - & - & - & - & - & - \\
	Offshore Wind & - & - & - & - & - & - \\
	Onshore Wind & - & - & - & - & - & - \\
	Conventional Nuclear & - & - & - & - & - & - \\
	Advanced Nuclear & - & - & - & - & - & - \\
	Small Modular Nuclear & - & - & - & - & - & - \\
	Batteries & - & - & - & - & - & - \\
	Pumped storage & - & - & - & - & - & - \\
	\bottomrule
\end{tabular}
\end{table}


\begin{table}[ht]
\caption[Technology capacity per Scenario S2]{Technology capacity per Scenario S2}
\labtab{technology_capacity_scenarios_s2}
\begin{tabular}{ c c c c c c c }
	\toprule
	Technologies & S2a & S2b & S2c & S2d & S2e & S2f \\
	\midrule
	Solar & - & - & - & - & - & - \\
	Offshore Wind & - & - & - & - & - & - \\
	Onshore Wind & - & - & - & - & - & - \\
	Conventional Nuclear & - & - & - & - & - & - \\
	Advanced Nuclear & - & - & - & - & - & - \\
	Small Modular Nuclear & - & - & - & - & - & - \\
	Batteries & - & - & - & - & - & - \\
	Pumped storage & - & - & - & - & - & - \\
	\bottomrule
\end{tabular}
\end{table}


\begin{table}[ht]
\caption[Technology capacity per Scenario S3]{Technology capacity per Scenario S3}
\labtab{technology_capacity_scenarios_s3}
\begin{tabular}{ c c c c c c c c c }
	\toprule
	Technologies & S3a & S3b & S3c \\
	\midrule
	Solar & - & - & -  \\
	Offshore Wind & - & - & -  \\
	Onshore Wind & - & - & -  \\
	Conventional Nuclear & - & - & -  \\
	Advanced Nuclear & - & - & -  \\
	Small Modular Nuclear & - & - & -  \\
	Batteries & - & - & -  \\
	Pumped storage & - & - & -  \\
	\bottomrule
\end{tabular}
\end{table}

\begin{table}[ht]
\caption[Technology capacity per Scenario S4]{Technology capacity per Scenario S4}
\labtab{technology_capacity_scenarios_s4}
\begin{tabular}{ c c c c c c c }
	\toprule
	Technologies & S4a & S4b & S4c & S4d & S4e & S4f \\
	\midrule
	Solar & - & - & - & - & - & - \\
	Offshore Wind & - & - & - & - & - & - \\
	Onshore Wind & - & - & - & - & - & - \\
	Conventional Nuclear & - & - & - & - & - & - \\
	Advanced Nuclear & - & - & - & - & - & - \\
	Small Modular Nuclear & - & - & - & - & - & - \\
	Batteries & - & - & - & - & - & - \\
	Pumped storage & - & - & - & - & - & - \\
	\bottomrule
\end{tabular}
\end{table}

\section{The Digest}


\begin{kaoboxgreen}[frametitle=Main Takeaways]

\begin{itemize}
\item The world consumes today around 160,000 TWh of energy, mostly from fossil fuels
\item Electrifying the world is a goal we should all have in mind. This implies increasing the energy consumption to around 200,000 TWh.
\item Electrification of the transport and heating sector would save primary energy, and we can imagine that this would drop the requirements by around one third.
\end{itemize}
  
\end{kaoboxgreen}