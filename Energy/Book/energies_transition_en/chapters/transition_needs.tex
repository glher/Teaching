\setchapterpreamble[u]{\margintoc}
\chapter{The Transition Needs}
\labch{transition_needs}

In this chapter I will assess two main scenarios. The first one assumes that the world of 2050 and later looks very similar to the world of today. That is, energy poverty, and by extension poverty, is not solved and the needs do not increase. We even look at an optimistic energy need, where the fossil fuels heat conversion is replaced by a more effective use of electricity, lowering the energy needs. A second scenario is also shown where we consider everyone on the planet to have a reasonable access to energy, with every country consuming at the Eastern European rate.

\begin{kaobox}[frametitle=Transition Scenarios]
\begin{itemize}
\item Scenario EOSP

The scenario Energy Optimist, Society Pessimist consider that the current energy needs only are transitioned. This means that the billions of people currently in energy poverty stay in that position for the next century.

\item Scenario EPSO

The scenario Energy Pessimist, Society Optimist consider that the world will see the electrification of most people living in poverty today, at a level  around 30,000 kWh or energy consumed per year per capita. As a reference point, this is roughly the situation of Poland, Greece or Portugal.
\end{itemize}
\end{kaobox}

Scenario EOSP indicates the need to transition 160,000 TWh, while scenario EPSO would have us transition 200,000 TWh.


\todo{Clarify GWe versus GWth}


When the media discuss energy transition, I feel that they often misrepresent, or don't explain enough, what they mean. There is a very large difference between electricity and energy. Electricity encompasses only the electric power uses, such as what allows a home outlet to charge a phone, plug a fridge, etc. Energy consists of all of the electricity as well as transportation and heating.

When using gas to power a car or to heat a location, a lot of the energy is lost in the process. Electricity would allow you to be more efficient. Consequently, the electrification of the transportation and thermal sector implies a better efficiency and less consumption, translating to a lower power needed.

In the following, we will make two things distinct:


\begin{kaobox}[frametitle=Electricity Transition]
Electricity transition is the removal of all energy from fossil fuels from the electrical grid. In other words, the gasoline powered cars can stay, the coal and gas plants have to go.
\end{kaobox}

\begin{kaobox}[frametitle=Energy Transition]
Energy transition is the removal of all energy from fossil fuels, to be replaced with non carbon-emitting sources.
\end{kaobox}

Electricity transition is not energy transition.


\section{Current Energy Sources}

The data for the current energy generation mix in the world is obtained.

[Placeholder for energy donut graph]

[Placeholder for energy time series graph]

[Placeholder for zoom on renewable share graph]




Energy use:

Primary energy includes heat conversion. The IEA gives estimates of the total final consumption~\sidecite{iea2020statistics}, in millions of tons of oil equivalent (MTOE). In an all electric world, this is what matters. From that, let's try and see what cannot be transitioned for now.

The fossil fuels consumption to replace is consequently 1000 MTOE for coal, 4000 MTOE of Oil, and 1500 MTOE of Natural Gas. Coal is used at 30\% for iron and steel production, at 10\% for petrochemical uses, and 20\% for non-metallic minerals, with another 5\% for non-energy use. The transition pathways for those applications seem difficult to achieve soon. The rest of the coal consumption, roughly 35\%, could be transitioned. Oil is used for industrial consumption (plastics notably) and for non-energy applications such as lubricants and other products. We can assume that 75\% of the energy consumed from oil could be transitioned, notably from the transportation sector. 12\% of natural gas is used for non-energy purposes, and we can approximate that 85\% could be transitioned to clean energy.

%Coal: 1000 MTOE
%--> 30\% iron and steel
%--> 10\% petrochemical
%--> 20\% non-metallic minerals
%--> 5\% non-energy use
%--> Rest (35\%) could transition

%Oil: 4000 MTOE
%--> 16\% non energy (lubricants etc)
%--> 8\% industrial (plastics etc)
%--> Rest (75\%) could transition

%Gas: 1500 MTOE
%--> 12\% non energy
%--> Rest (85\%) could transition

So, we could transition around $0.35 * 1000 + 0.75 * 4000 + 0.85 * 1500 = 4625 MTOE$. This represents 70\% of the fossil fuels consumption today. Note that fossil fuels account for around 70\% of the total energy consumption themselves.

%0.35 * 1070 + 0.75 * 3985 + 0.85 * 1502 = 4640 out of 6557 MTOE, or 70%

%In 2017, fossil fuels accounted for 6557 MTOE out of 9717 MTOE, or around 70\%


The world today uses ~160,000 TWh of equivalent primary energy. It breaks down, roughly to:

\begin{itemize}
\item Oil: 55,000 TWh (primary)
\item Natural Gas: 40,000 TWh (primary)
\item Coal: 45,000 TWh (primary)
\item Nuclear: 7,000 TWh (electricity)
\item Hydro: 10,000 TWh (electricity)
\item Renewable: 8,000 TWh (electricity)
\end{itemize}

In the case of fossil fuels, we see that the primary energy consumed is given. This is due to the fact that there is a penalty to using heat engine, as it's not a very efficient process, and approximately two third of the primary energy is lost in the process of converting to electricity. We want to assess the needs in a 100\% electrified world. This means that we can get rid of the heat efficiency issue for fossil fuels notably and replace with a more efficient electricity use.

We have seen above that we can imagine replacing around 70\% of our fossil fuels needs with electricity. This represents around 100,000 TWh, distributed across three main sectors: transportation, electricity, and direct heat. No change in energy consumption is to be expected from the direct heat sector, and renewables will have trouble making a dent in that sector. It consists mostly of industrial processes requiring very high temperatures. Biomass or waste fuels could be used, but could encounter a not-renewable-anymore issue.

Transportation represents around a third of the fossil fuel use. The efficiency of a modern gasoline car is around 30\%. The efficiency of an EV is around 75\%. In order to replace all the gasoline used in transportation, we need to replace approximately 50,000 TWh of fossil fuel (oil). If our EVs were 100\% efficient at converting electricity to power, we would need 30\% of the 50,000 TWh, or 15,000 TWh. Since they are around 75\% efficient, our actual need is 15,000 TWh divided by 75\%, or 20,000 TWh\sidenote[][-2mm]{This represents a heat rate conversion factor of 40\%. You may find that number when people compare primary energy between fossil fuels and renewables.}. Consequently, if the entire transportation sector were to switch to electric vehicles, the energy requirements would drop from 50,000 TWh to 20,000 TWh, or by 60\%.

Electricity represents around another third of the fossil fuels uses. For the power plants part, we can assume a relatively similar heat rate to convert. We do have a higher efficiency in a turbine than in a gasoline car, but we also have a higher efficiency of our renewable energy production system than batteries trips. Here we want to transition approximately 50,000 TWh again. So we get, using our 40\% efficiency conversion factor, a drop in our energy requirements from 50,000 TWh to 20,000 TWh again, or another 60\% drop.


From these first order estimates, we conclude that maximizing the electrification of the energy grid would mean that we go from around 160,000 TWh to 100,000 TWh, or a 35 to 40\% drop, all else being equal (EOSP scenario). In the EPSO scenario, the needs would increase, and consequently we would have to meet 130,000 TWh via electricity.


It's important to note that those estimates are very optimistic, both in terms of actual transition feasibility and in terms of future energy needs. It assumes the transition of the entire world transportation sector (so, not one gasoline powered car left), and of most things that can technically transition to an electricity source.

%Assuming an extremely optimistic electrification of the world transport sectors and other heat engines, let us assume that only 150000 TWh of energy will be needed every year. Note that this is probably on the very, very low end of future needs.

\section{Current Carbon Emissions}

This section computes an estimate of the yearly carbon emissions in both scenarios assuming a business-as-usual situation. This will give us a good idea of what we need to avoid producing.


\section{Scenarios Pathway}

This section looks at all the installation we need for each scenario. It should be a table:



\begin{table}[ht]
\caption[Technology capacity per Scenario S1]{Technology capacity per Scenario S1}
\labtab{technology_capacity_scenarios_s1}
\begin{tabular}{ c c c c c c c }
	\toprule
	Technologies & S1a & S1b & S1c & S1d & S1e & S1f \\
	\midrule
	Solar & - & - & - & - & - & - \\
	Offshore Wind & - & - & - & - & - & - \\
	Onshore Wind & - & - & - & - & - & - \\
	Conventional Nuclear & - & - & - & - & - & - \\
	Advanced Nuclear & - & - & - & - & - & - \\
	Small Modular Nuclear & - & - & - & - & - & - \\
	Batteries & - & - & - & - & - & - \\
	Pumped storage & - & - & - & - & - & - \\
	\bottomrule
\end{tabular}
\end{table}


\begin{table}[ht]
\caption[Technology capacity per Scenario S2]{Technology capacity per Scenario S2}
\labtab{technology_capacity_scenarios_s2}
\begin{tabular}{ c c c c c c c }
	\toprule
	Technologies & S2a & S2b & S2c & S2d & S2e & S2f \\
	\midrule
	Solar & - & - & - & - & - & - \\
	Offshore Wind & - & - & - & - & - & - \\
	Onshore Wind & - & - & - & - & - & - \\
	Conventional Nuclear & - & - & - & - & - & - \\
	Advanced Nuclear & - & - & - & - & - & - \\
	Small Modular Nuclear & - & - & - & - & - & - \\
	Batteries & - & - & - & - & - & - \\
	Pumped storage & - & - & - & - & - & - \\
	\bottomrule
\end{tabular}
\end{table}


\begin{table}[ht]
\caption[Technology capacity per Scenario S3]{Technology capacity per Scenario S3}
\labtab{technology_capacity_scenarios_s3}
\begin{tabular}{ c c c c c c c c c }
	\toprule
	Technologies & S3a & S3b & S3c \\
	\midrule
	Solar & - & - & -  \\
	Offshore Wind & - & - & -  \\
	Onshore Wind & - & - & -  \\
	Conventional Nuclear & - & - & -  \\
	Advanced Nuclear & - & - & -  \\
	Small Modular Nuclear & - & - & -  \\
	Batteries & - & - & -  \\
	Pumped storage & - & - & -  \\
	\bottomrule
\end{tabular}
\end{table}

\begin{table}[ht]
\caption[Technology capacity per Scenario S4]{Technology capacity per Scenario S4}
\labtab{technology_capacity_scenarios_s4}
\begin{tabular}{ c c c c c c c }
	\toprule
	Technologies & S4a & S4b & S4c & S4d & S4e & S4f \\
	\midrule
	Solar & - & - & - & - & - & - \\
	Offshore Wind & - & - & - & - & - & - \\
	Onshore Wind & - & - & - & - & - & - \\
	Conventional Nuclear & - & - & - & - & - & - \\
	Advanced Nuclear & - & - & - & - & - & - \\
	Small Modular Nuclear & - & - & - & - & - & - \\
	Batteries & - & - & - & - & - & - \\
	Pumped storage & - & - & - & - & - & - \\
	\bottomrule
\end{tabular}
\end{table}

\section{The Digest}


\begin{kaoboxgreen}[frametitle=Main Takeaways]

\begin{itemize}
\item The world consumes today around 160,000 TWh of energy, mostly from fossil fuels
\item Electrifying the world is a goal we should all have in mind. This implies increasing the energy consumption to around 200,000 TWh.
\item Electrification of the transport and heating sector would save primary energy, and we can imagine that this would drop the requirements by around one third.
\end{itemize}
  
\end{kaoboxgreen}