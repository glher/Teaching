\setchapterpreamble[u]{\margintoc}
\chapter{Introduction}
\labch{intro}

\section{The Main Ideas}

Climate is rapidly changing and starting to impact society. This is only the beginning. While a lot of people have finally realized the dangers human society is in, too many carry an optimism that can be damaging.

The thought often encountered is that of the technology savior. That is, yes, climate change is a problem, but technology is going to find something and we will all be alright. I am not saying this is impossible. A century ago, who could have predicted the scale of the digital world, for example?

However, it is important to note that technology has to follow the laws of physics, and is competing in a inflexible human society. This text aims at helping people consider the big picture, and to not forget the scale and magnitude of the issues and solutions put forth.

Communication is a aspect of science which, I feel, has been degrading fast. I believe that this is due, in part, to the myriad of scientific fields and ultra specialized research, and in part to social media and jumping-to-conclusions journalism. The problem is that it causes people to not realize that success in several fields does not mean success as a whole. Climate change is a problem of physics, geophysics, computer science, material science, chemistry, energetic systems, engineering, economy, politics, social sciences, agriculture, and many more fields.

Only when the big picture is known by the majority of people can a plan be thought of and enacted efficiently. Until then, society is grasping at straws and potentially heading toward the wrong idea, or shutting down development of the right ones.

\section{What This Text Does}
\labsec{does}

This book focuses on the big pictures. You will not find fancy modeling with multi-line equations and strange Greek symbols and integrals or differential. Plenty of people do that already. Instead, you will be exposed to first-order thoughts pattern.

What this entails is while complex models are great and a useful tools, they often make flawed assumptions to the real world, especially when multiple scientific fields intersects, and especially when future predictions and human society are part of the equations. Some models\sidenote[][-2mm]{Global Circulation Models notably, also commonly known as Climate Models}, used as inputs later on, are extremely complex. But our goal in this book is to speak in terms of orders of magnitude.

Imagine you are very much in love and trying to save up for the wedding of your dreams in a couple years\sidenote[][-2mm]{A cheesy example, but bear with me}.

You could take a complex approach. You figure out exactly how many guests will be there, how many won't be able to come, how many will decide to cancel at the last minute, how many are vegetarian, how many will bring a plus one, etc. This gives you an optimized meal cost. Then you repeat the process for the DJ, finding a super cheap and awesome performer. You look online and find a venue that does a wine bar for a crazy low price, so you go ahed and book it for your date. You got it. Your dream wedding for \$12,543. You can even try and optimize your model so that you do not do an open bar in order to add more people to the party.

Fast forward a year or so, you are finally engaged! Congratulations! You're now looking into booking everything on your list. You know exactly how much it will cost after all, according to your complex model. But, calling up vendors, you see that they charge a bit more than they used to. Your awesome DJ is already booked for the date you want, you can only find one that is a lot more expensive now. The cheap venue you had booked went under and is not available anymore, but this other works, even if it's more expensive and not as good. A pandemic hits and everything shuts down\sidenote[][-2mm]{That one is pretty far-fetched, I know!}. Guest RSVPs and then cancel the day of. All in all, a reasonably common wedding experience. The days after, you realize that you actually paid \$19,845.

Your nice, well-thought of, complex model failed you. So would a simple model have helped you save your perfect wedding? No. But it would have made you more prepared for the actual costs by taking a range of potential values, and you may have gotten an approximate value of \$18,000. You would have been better prepared and known that you really could do without the ice sculptures.

Of course, this is not an adequate metaphor for climate change action. At the end of this little story, you still get married, and you still end up happy. The end. I am afraid the situation is a little more bleak when climate change comes into play.

Having said that, this book does aim at giving you an idea of the scale of the problem, and why articles that say that we have the solutions, if only we could just implement them, are most of the times vastly underestimating the magnitudes or misunderstanding the scientific assumptions and shortcomings of the article they report on.

So, we will start by looking back at history, so that we can realize how fantastic fossil fuels are. Once we have done that, we will look at the future, so that we can realize how terrible fossil fuels are. And we will see what various paths forward could mean to the world, assessing the truth of the various energetic transition claims we often see in the news from government or large companies. Finally, from all of that, we will try to assess what any individual can do to help and what one can reasonably expect.

\section{What This Text Does Not Do}
\labsec{doesnot}

This book does not advocate for or criticize any specific technologies for fun. I may seem that way, as you will see that some ideas unfortunately seem to be shot down by reality, though they are great in theory. Of course, in our demonstrations, we will use only the current known scientific theories, and push it further with optimistic scenarios. We will not count on any timely breakthrough, and we will not consider technologies that are not even being thought of today.

Again, our goal is to stand at the frontier between the real world\sidenote[][*-2]{A real world, with real people, most of which do not even have a device to read this text}  and the scientific specialty world\sidenote[][]{As the physicists often say, let us assume a spherical cow...}.

We will not explore the technologies in details, as they would and have each taken entire volumes already, but we will briefly recall the basic principles. This will be necessary to show the physical and societal limitations one may run into sometimes. 


