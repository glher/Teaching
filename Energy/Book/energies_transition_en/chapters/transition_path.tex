\setchapterpreamble[u]{\margintoc}
\chapter{The Transition Path}
\labch{transition_path}

In this chapter I use the two preceding chapters, on economics, carbon, and issues to draw conclusion on the best, more logical, path forward. I point again that having complex "flux tendu" models that meet the criteria for some countries does not make the problems go away. When the scientific community so strongly disagree, it is wise to err on the side of caution and aim for the least stringent path. If the 100\% renewable camp is wrong and fails to overcome the hurdles, the plan fail. If the "mixed energies, with nuclear" camp is wrong, the plan is still alright. The logical choice is consequently pretty self evident.


\blindtext



\section{Risk Matrix}

We develop a color-coded qualitative risk matrix, with the rows being the scenario, and the columns being the section issue. For example, for S3, nuclear siting is red and uranium is orange, and for S1, copper is orange, RRE is black, Lithium is black, pumped storage is black, etc.

For some technologies, and some materials, the physical limits really seem too large. It is difficult to see how the Dysprosium, Neodymium, Lead or Lithium resource problems could be overcome without a major breakthrough defying the odds. This tends to disqualify batteries system, and consequently severely dampen hopes for a 100\% renewable price, even if the economical impact derived in \vrefsec{renewable_problem} was considered a non-factor.

Copper is likely to become a severe problem over time, though it may not be directly due to the renewable industry. It seems to be within the realm of doable over the next century, even in a full renewable world (though with no batteries, which also use a lot of copper).

A field to seriously explore and develop is the recycling of those materials. Some recycling chains exist currently, at a relatively modest scale. Inherently, with the mines drying up, the cost will go up and make recycling more prevalent\sidenote[][-2mm]{though that also mean a likely cost increase, potentially pricing out some technologies for wide uses}.

\begin{kaobox}[frametitle=TO-DO]
Is there a path around the materials problem? New materials, interesting research, \ldots ?
\end{kaobox}

\section{What does this mean?}

What appears to be the best solution? Is there a choice between multiple scenarios, or is the real, actual, choice only on the share in the mix? Discuss the implications.


\section{The Digest}


\begin{kaoboxgreen}[frametitle=Main Takeaways]

\begin{itemize}
\item This has not been done yet
\item Reading this will teach you absolutely nothing
\item I am serious, I could type random letters and it would give you as much information
\item Fedhiz gavartz hedtz inewps
\end{itemize}
  
\end{kaoboxgreen}