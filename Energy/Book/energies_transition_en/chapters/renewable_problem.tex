\setchapterpreamble[u]{\margintoc}
\chapter{A Renewable Problem}
\labch{renewable_problem}

In this chapter I will go over the approximate costs incurred in each of the nine scenario we have defined, listed on the side.

We will pull the data we have described in \refch{transition_needs} and \refch{technology_costs} and apply the reasoning derived in \refch{renewable_solution}.


\begin{kaobox}[frametitle=Scenarios to consider]

\begin{itemize}
	\item S1: Onshore Wind -- Pumped Hydro Storage
	\item S2: Onshore Wind -- Batteries
	\item S3: Offshore Wind -- Pumped Hydro Storage
	\item S4: Offshore Wind -- Batteries
	\item S5: Solar -- Pumped Hydro Storage
	\item S6: Solar -- Batteries
	\item S7: Nuclear
	\item S8: Renewable Mix -- Storage Mix
	\item S9: Nuclear -- Renewable Mix
\end{itemize}

Each of the scenarios will furthermore be tested for sensitivity by modifying parameters and looking at a range of future potentials, from pessimistic to optimistic depending on the energy source.

\end{kaobox}

\section{Electricity Transitions}

First, we calculate an estimate of the costs per scenario, for each country, in \vreftab{scenario_costs_estimates}. This shows that things get very expensive, and that some options are clearly more efficient economically than others. \vreftab{scenario_costs_ratio} shows the normalization relative to the least expensive scenario each time\sidenote[][-2mm]{Nuclear, with a score of 1}. Even though in our case studies we do not vary the prices locally, and assume that a project will cost the same in every country, we see that the ratios vary. This is mostly due to the effect of changing load factors. Not every country is equal when it comes to natural resources, and wind and solar are natural resources.


\begin{table}[ht]
\caption[Scenario Costs Estimate over 100 years per country (Trillions US Dollars -- Electricity only -- 7 days of storage)]{Scenario Costs Estimate over 100 years per country (Trillions 2020 US Dollars) -- Electricity only -- 7 days of storage -- Backup Factor}
\labtab{scenario_costs_estimates}
\begin{tabular}{ c c c c c c c c c c }
	\toprule
	Country & S1 & S2 & S3 & S4 & S5 & S6 & S7 & S8 & S9 \\
	\midrule
	France & 1.9 & 9.1 & 3.2 & 8.7 & 2.7 & 17.2 & 1.2 & S8 & S9\\
	USA & 12.2 & 71.4 & 17.7 & 62.1 & 15.2 & 133.5 & 9.1 & S8 & S9\\
	Brazil & 1.4 & 9.6 & 2.4 & 8.6 & 2.2 & 18.6 & 1.2 & S8 & S9\\
	China & 29.2 & 128.8 & 40.8 & 115.5 & 31.5 & 230.8 & 15.2 & S8 & S9\\
	Nigeria & 0.08 & 0.46 & 0.11 & 0.4 & 0.09 & 0.85 & 0.06 & S8 & S9\\
	Nigeria\sidenote[$\dagger$][-2mm]{With access to an average French person electricity}  & 4.2 & 23.1 & 5.6 & 19.8 & 4.7 & 42.5 & 2.9 & S8 & S9\\
	\bottomrule
\end{tabular}
\end{table}


\begin{table}[ht]
\caption[Scenario Costs Estimate over 100 years per country (Trillions US Dollars -- Electricity only -- 7 days of storage)]{Scenario Costs Estimate over 100 years per country (Trillions 2020 US Dollars) -- Electricity only -- 7 days of storage}
\labtab{scenario_costs_estimates}
\begin{tabular}{ c c c c c c c c c c }
	\toprule
	Country & S1 & S2 & S3 & S4 & S5 & S6 & S7 & S8 & S9 \\
	\midrule
	France & 2.5 & 20.6 & 3.9 & 22.0 & 2.9 & 21.0 & 1.2 & 2.0 & 1.7\\
	USA & 12.2 & 71.4 & 17.7 & 62.1 & 15.2 & 133.5 & 9.1 & S8 & S9\\
	Brazil & 1.4 & 9.6 & 2.4 & 8.6 & 2.2 & 18.6 & 1.2 & S8 & S9\\
	China & 29.2 & 128.8 & 40.8 & 115.5 & 31.5 & 230.8 & 15.2 & S8 & S9\\
	Nigeria & 0.08 & 0.46 & 0.11 & 0.4 & 0.09 & 0.85 & 0.06 & S8 & S9\\
	Nigeria\sidenote[$\dagger$][-2mm]{With access to an average French person electricity}  & 4.2 & 23.1 & 5.6 & 19.8 & 4.7 & 42.5 & 2.9 & S8 & S9\\
	\bottomrule
\end{tabular}
\end{table}


\begin{table}[ht]
\caption[Scenario Costs Ratios Estimate over 100 years per country (Trillions US Dollars -- Electricity only -- 7 days of storage)]{Scenario Costs Ratios Estimate over 100 years per country (Trillions 2020 US Dollars) -- Electricity only -- 7 days of storage}
\labtab{scenario_costs_estimates}
\begin{tabular}{ c c c c c c c c c c }
	\toprule
	Country & S1 & S2 & S3 & S4 & S5 & S6 & S7 & S8 & S9 \\
	\midrule
France & 1.6 & 7.6 & 2.7 & 7.3 & 2.3 & 14.3 & 1 & S8 & S9\\
USA & 1.3 & 7.8 & 1.9 & 6.8 & 1.7 & 14.7 & 1 & S8 & S9\\
Brazil & 1.2 & 8.0 & 2.0 & 7.2 & 1.8 & 15.5 & 1 & S8 & S9\\
China & 1.9 & 8.5 & 2.7 & 7.6 & 2.1 & 15.2 & 1 & S8 & S9\\
Nigeria & 1.3 & 7.7 & 1.8 & 6.7 & 1.5 & 14.2 & 1 & S8 & S9\\
$\mbox{Nigeria}^{\dagger}$ & 1.4 & 8.0 & 1.9 & 6.8 & 1.6 & 14.7 & 1 & S8 & S9\\
	\bottomrule
\end{tabular}
\end{table}







%\sidenote[\dagger][-2mm]{With access to an average French person electricity} 
Another way to express it is as a ratio value. How much more expensive than the minimum cost over all considered scenario?

\blindtext


\section{Energy Transitions}

We do the same tables and figures here but for energy as a whole.


\blindtext


\section{Sensitivity Analysis}

This section models the impact of changing the prices, increasing or decreasing the demand, modifying the storage duration.

\blindtext

\section{The Digest}


\begin{kaoboxgreen}[frametitle=Main Takeaways]

\begin{itemize}
\item All the scenarios are compared in terms of absolute estimated costs and relative costs
\item The goal is not only to transition to a clean energy environment, but to transition to a long-term sustainable environment
\item Nuclear is the cheapest option when looking at a reasonable time frame, and a sensitivity analysis shows that it would take a lot of things going against markets expectations to be more cost efficient
\item Batteries are the biggest economic hurdles for renewable systems
\end{itemize}
  
\end{kaoboxgreen}



